%% file `template_en.tex‘, modern cv
%% Copyright 2006-1008 Xavier Danaux (xdanaux@gmail.com).
%
% This work may be distributed and/or modified under the
% conditions of the LaTeX Project Public License version 1.3c,
% available at http://www.latex-project.org/lppl/.


\documentclass[11pt,a4paper]{moderncv}

% moderncv themes
\moderncvtheme[blue]{classic}                 % optional argument are 'blue' (default), 'orange', 'red', 'green', 'grey' and 'roman' (for roman fonts, instead of sans serif fonts)
%\moderncvtheme[green]{classic}                % idem

% character encoding
\usepackage[utf8]{inputenc}                   % replace by the encoding you are using
\usepackage{pslatex}


% adjust the page margins
\usepackage[scale=0.8]{geometry}
%\setlength{\hintscolumnwidth}{3cm}						% if you want to change the width of the column with the dates
%\AtBeginDocument{\setlength{\maketitlenamewidth}{6cm}}  % only for the classic theme, if you want to change the width of your name placeholder (to leave more space for your address details
\AtBeginDocument{\recomputelengths}                     % required when changes are made to page layout lengths
\makeatletter
\def\url@leostyle{%
  \@ifundefined{selectfont}{\def\UrlFont{\sf}}{\def\UrlFont{\small\rmfamily}}}
\makeatother
%% Now actually use the newly defined style.
\urlstyle{leo}


% personal data
\firstname{Jitao David}
\familyname{Zhang}
\title{\fontsize{13}{15}\selectfont Computational Biology und Biostatistik}               % optional, remove the line if not wanted
\address{Im Neuenheimer Feld 580}{69120 Heidelberg}    % optional, remove the line if not wanted
\mobile{+49-(0)176-5325-5754}                    % optional, remove the line if not wanted
\phone{+49-(0)6221-42-3458}                      % optional, remove the line if not wanted
\fax{+49-(0)6221-42-3454}                          % optional, remove the line if not wanted
\email{jitao\_david.zhang@roche.com}                      % optional, remove the line if not wanted
%\extrainfo{\url{http://www.NextBioMotif.com}} % optional, remove the line if not wanted
\photo[85pt]{CVfoto0}                         % '64pt' is the height the picture must be resized to and 'picture' is the name of the picture file; optional, remove the line if not wanted
% \quote{Some quote (optional)}                 % optional, remove the line if not wanted

%\nopagenumbers{}                             % uncomment to suppress automatic page numbering for CVs longer than one page


%----------------------------------------------------------------------------------
%            content
%----------------------------------------------------------------------------------
\begin{document}
\renewcommand*{\namefont}{\fontsize{30}{32}\sffamily\mdseries\upshape}
\maketitle

%\section{Stichwort}
%\cvlistdoubleitem[+]{Computational Biology}{Biostatistik}

%\section{Education}
\section{Ausbildung}
\cventry{2008--2011}{Promotion}{Abteilung Molekulare
  Genomanalyse, Deutsches Krebsforschungszentrum}{Heidelberg}{Bis Januar 2011}{Thema:
  \textquotedblleft\emph{Computational biology and Biostatistical Approaches to
    Profile Cancer-related Gene Regulatory Network with Integrated
    High-throughput Data}\textquotedblright. Betreuer: PD Dr. Stefan
  Wiemann.\newline{}
  Die Doktorarbeit konzentriert sich auf die Integration der
  statistischen Analysen, mathematischen Methoden und Pathway-Analyse
  im Rahmen der Molekularen Onkologie, um neue diagnostische Marker und
  therapeutische Methode gegen Krebs \textit{in silico} zu
  identifizieren. Als Expert über R und mit fundierten Kenntnissen in Biologie entwickelte ich mehrere Software-Pipelines, die statistische Modelle und
  neuartige Algorithmen implementieren, um die \textit{omics-}Daten
  aus heterogenen Quellen, drunter Hochdurchsatz RNAi/Compound
  Screening (HTS), Microarray/NGS, klinische
  Studien sowie öffentliche Datenbanken, zu analysieren. Zusätzliche Know-How
  umfasst Netzwerkanalyse und multivariate Verfahren, dazu ausgezeichnete
  Selbständigkeit, Teamgeist und Projektmanagement.}
\cventry{2007--2008}{EU Marie-Curie Fellow}{European Bioinformatics
  Institute}{Cambridge, UK}{}{Computational-Biology Forschung bei
  Dr. Wolfgang Huber mit dem Thema \textit{Epistatic genetic interactions in yeast using Affymetrix tiling array.}}{}
\cventry{2006--2008}{M.Sc. Biologie/Bioinformatik}{Universität Heidelberg}{}{\textit{Note: Sehr Gut}}{Abschlussarbeit: \textquotedblleft\emph{Data analysis of a genome-wide RNAi screen to identify modulators of human p38 MAP kinase}\textquotedblright. Stichwort: \emph{statistische Datenanalyse}, \emph{siRNA High-Throughput-Screening}}
\cventry{2002--2006}{B.Sc. Biologie, mit Auszeichnung}{Peking Universität}{Beijing,China}{\textit{GPA: 3,73 von 4,00}}{}  % arguments 3 to 6 are optional
\cventry{1999--2002}{Abitur}{Tianjin Nr.1 Gymnasium}{Tianjin,China}{\textit{711/750 Punkten}, bestes Abitur der Stadt}{}  % arguments 3 to 6 are optional
%% B.Sc.: Mit Auszeichnung, Ranking top 5\%. Nebenfach: Germanistik, Ranking 1/30
%\section{Promotion}
%\cvline{Thema}{\emph{}}
%\cvline{Betreuer}{PD Dr. Stefan Wiemann, Abteilung Molekulare Genomanalyse, DKFZ}
%\cvline{Thesis Advisory Commitee}{Dr. Wolfgang Huber, Dr. Rainer K\"onig}
%\cvline{Kurzbeschreibung}{\small}

\section{Praktika und Berufstätigkeit}
\cventry{2006--2008}{Bioinformatik Programmierung}{HUSAR Bioinformatik
  Gruppe, DKFZ}{Heidelberg}{}{Programmierung mit Bezug auf
  Bioinformatik hauptsächlich in Perl, C und C++ als Wissenschaftliche
  Hilfs-kraft. Stichwort: Entwicklung und Debugging, Unit-Testing, XML, UML-Modellierung.}                % arguments 3 to 6 are optional

%\section{Awards and Fellowship}
\section{Auszeichnungen und Stipendien (Auswahl)}
\cventry{2008--2011}{Stipendium durch \emph{DKFZ International PhD Program}}{Heidelberg}{}{}{}
\cventry{2007--2008}{\emph{Marie-Curie Fellowship} der EU}{Cambridge, UK}{}{}{}
\cventry{2006--2007}{Stipendium durch \emph{ZMBH-DKFZ International
    MCB Master Program}}{Heidelberg}{}{}{}
%\cventry{2004--2005}{\emph{Nationales Stipendium}}{zweifach Erster Klasse, Beijing, China}{}{}{}


% Publications from a BibTeX file
\nocite{*}
\renewcommand\refname{Publikationen (Auswahl)}
\bibliographystyle{plain}
\bibliography{mypubs}       % 'publications' is the name of a BibTeX file

\section{Software (Auswahl)}
\cventry{KEGGgraph}{R/Bioconductor}{Netwerkanalyse der biologischen
  Pathways mit Graphentheorie}{}{}{}
\cventry{ScratchIt}{C++}{Robuste Analyse der Wound-Healing Assays
  mit High-Content-Mikroskopie}{}{}{}
\cventry{flowDeconvolutor}{R/Bioconductor}{Statistische Analyse
  der Zellzyklus-Experimente mit FACS}{}{}{}
%\cventry{RTCA}{Bioconductor Software}{Analysesoftware für das Real-Time
%  Cell-Analyzer System}{}{}{}


%\section{Talks}
%\section{Vortr\"age}
%\cventry{November 2008}{RpsiXML: Bridging PSI-MI XML standard with statistical and computational environment of R}{Workshop on the Development of Standards-compliant tools for molecular interaction data management, Cambridge, UK}{}{}{}

\section{Poster und Publikationen ohne Peer-Review (Auswahl)}
\cventry{November 2010}{Catching the dynamics of RNAi phenotypes
  during a human kinome screen}{System Genomics 2010, Heidelberg,
  Germany}{\textbf{JD. Zhang}, Udo Eichenlaub, \textit{et al.}}{}{}
\cventry{Juni 2009}{A novel large-scale screen to identify modulators
  of miR-21}{MicroRNA and Cancer Meeting 2009, Colorado,
  USA}{\textbf{JD. Zhang}, I. Keklikoglou, \textit{et al.}}{}{}
\cventry{Februar 2009}{Analyzing the Effect of Gene Knock-Down in
  Real-Time Using Roche's xCELLigence System}{Biochemica
  2009/02}{\textbf{JD. Zhang}, A. Duda, \textit{et al.}}{}{}
%\cventry{Februar 2009}{qPCR Identification of Genes Involved in Apoptosis and Cell Cycle Regulation}{Biochemica 2009/02}{\textbf{JD. Zhang}, U. Ernst, A. Irsigler, S. Wiemann and U. Tschulena}{}{}
%\cventry{Mai 2008}{DNAdeconvolutor: software for DNA histogram analysis in R and Bioconductor}{Systems Genomics Meeting 2008, Heidelberg}{\textbf{JD. Zhang}, S. Bechtel, S. Wiemann and A. Poustka}{}{}
%\cventry{August 2007}{Protein Interaction Pool (PIP) -- A comprehensive database and interface to protein interaction data}{ISCB 2007, Wien}{J. Kunert, T. Beissbarth, \textbf{JD. Zhang}, A. Poustka, S. Wiemann and A. Mehrle}{}{}

\renewcommand{\listitemsymbol}{-} % change the symbol for lists
%\section{Current projects}
%\section{Aktuelle Projekte}
%\cvlistitem{Impact of microRNA regulation on biological pathways}
%\cvlistitem{Cell-based screening to identify regulators of human microRNA miR-21 (Kooperation mit Dr. Ulrich Tschulena)}
%\cvlistitem{New cell-based screening with multi-demensional output to identify novel regulators of ERBB signaling system (Kooperation mit Dr. \"Ozg\"ur Sahin)}


%\section{Computer skills}
\section{Computerkenntnisse}
\cvcomputer{Programmierung}{R/Bioconductor, C, C++ , Java}{Datenbank}{MySQL, MSSQL}%{Datenbank}{MySQL, MSSQL}
\cvcomputer{Skriptsprachen}{Python, Bash, Perl}{Web-Produktion}{J2EE (JSF), Groovy and Grails, GWT}
\cvcomputer{Betriebssystem}{Debian Linux, Solaris, Windows}{Andere}{XML, OpenCV, ImageMagick, \LaTeX{}}
%\cvcomputer{category 3}{XXX, YYY, ZZZ}{category 6}{XXX, YYY, ZZZ}

%\section{Languages}
\section{Sprachkenntnisse}
\cvlanguage{Deutsch}{Konversationssicher}{Vier Jahren in Deutschland gelebt}
\cvlanguage{Englisch}{Verhandlungssicher}{6-Monate Fellowship in Cambridge, UK}
\cvlanguage{Japanisch}{Grundkenntnisse}{}
%%\cvlanguage{Chinesisch}{Muttersprache}{}

%\section{Interests}
\vspace*{-1em}
\section{Engagement}
\cventry{2009--\hspace{4.25ex}}{Coaching R-Programmierung für Einsteiger und Fortgeschrittene in DKFZ}{}{}{}{}% arguments 3 to 6 are optional
\cventry{2007--\hspace{4.25ex}}{Mitglieder des Fachverbandes \emph{R Foundation} für das Statistik-Programm \emph{R}}{}{}{}{}% arguments 3 to 6 are optional
%\cventry{2003--\hspace{4.25ex}}{Ehrenamtliches Vorstandmitglied des Vereins \emph{GerFans China}}{}{}{}{}% arguments 3 to 6 are optional
%\cventry{2004--2006}{Fussballjournalist und Redaktionsassistent}{KICKER Chinas}{Beijing}{}{}
%%Kolumne \emph{Die brillianten 40 Jahre der Bundesliga} für die Yangtze Abendzeitung.}% arguments 3 to 6 are optional
%\cventry{2002--2006}{Deutsch-\"Ubersetzer für chinesische Zeitschriften über Klassische Musik}{}{}{}{}% arguments 3 to 6 are optional

\vspace*{-1em}
\section{Personalien}
\cvcomputer{Geburt}{28.09.1983 in Tianjin, China}{Nationalität}{Chinesisch}
\cvcomputer{Familienstand}{Verheiratet mit Ying Guo seit Dec. 2010}{}{}
\end{document}

%\renewcommand{\listitemsymbol}{-} % change the symbol for lists

%\section{Extra 1}
%\cvlistitem{Item 1}
%\cvlistitem{Item 2}
%\cvlistitem[+]{Item 3}            % optional other symbol

%\section{Extra 2}
%\cvlistdoubleitem[\Neutral]{Item 1}{Item 4}
%\cvlistdoubleitem[\Neutral]{Item 2}{Item 5}
%\cvlistdoubleitem[\Neutral]{Item 3}{}






%% end of file `template_en.tex'.
