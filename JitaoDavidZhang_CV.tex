%% file `template_en.tex‘, modern cv
%% Copyright 2006-1008 Xavier Danaux (xdanaux@gmail.com).
%
% This work may be distributed and/or modified under the
% conditions of the LaTeX Project Public License version 1.3c,
% available at http://www.latex-project.org/lppl/.


\documentclass[11pt,a4paper]{moderncv}

% moderncv themes
\moderncvtheme[blue]{classic}                 % optional argument are 'blue' (default), 'orange', 'red', 'green', 'grey' and 'roman' (for roman fonts, instead of sans serif fonts)
%\moderncvtheme[green]{classic}                % idem

% character encoding
\usepackage[utf8]{inputenc}                   % replace by the encoding you are using
\usepackage{pslatex}
\usepackage{textcomp}

% adjust the page margins<
\usepackage[scale=0.8]{geometry}
%\setlength{\hintscolumnwidth}{3cm}						% if you want to change the width of the column with the dates
%\AtBeginDocument{\setlength{\maketitlenamewidth}{6cm}}  % only for the classic theme, if you want to change the width of your name placeholder (to leave more space for your address details
\AtBeginDocument{\recomputelengths}                     % required when changes are made to page layout lengths
\makeatletter
\def\url@leostyle{%
  \@ifundefined{selectfont}{\def\UrlFont{\sf}}{\def\UrlFont{\small\rmfamily}}}
\makeatother

%% Now actually use the newly defined style.
\urlstyle{leo}


% personal data
\firstname{Jitao David}
\familyname{Zhang}
\title{\fontsize{12}{14}\selectfont A Computational Biologist in Drug Discovery}               % optional, remove the line if not wanted
\address{F. Hoffmann-La Roche AG}{4070 Basel, Switzerland}    % optional, remove the line if not wanted
%\mobile{+49-(0)176-5325-5754}                    % optional, remove the line if not wanted
%\phone{+41-(0)6168-86251}                      % optional, remove the line if not wanted
%\fax{+49-(0)6221-42-3454}                          % optional, remove the line if not wanted
\homepage{jdzhang.me}
\email{jitao\_david.zhang@roche.com}
\photo[85pt]{CVfoto0}                         % '64pt' is the height the picture must be resized to and 'picture' is the name of the picture file; optional, remove the line if not wanted
%\quote{Carpe Diem}                 % optional, remove the line if not wanted

%\nopagenumbers{}                             % uncomment to suppress automatic page numbering for CVs longer than one page


%----------------------------------------------------------------------------------
%            content
%----------------------------------------------------------------------------------
\begin{document}
\renewcommand*{\namefont}{\fontsize{32}{32}\sffamily\mdseries\upshape} 
%\renewcommand{\namefont}{\fontsize{30}{32}\sffamily\mdseries\upshape}
\maketitle

\section{Key facts}
\cvlistitem{Interdisciplinary researcher with more than ten years' experience in drug discovery.}
\cvlistitem{Supporting 30+ projects per year by algorithm development and data modelling.}
\cvlistitem{Open-source software developer and open-access author (40+ publications, \textit{h}-index 24).}
\cvlistitem{Proponent of multiscale modelling of drug mechanism and safety.}

\section{Research career}
\cventry{2011--\hspace{4.25ex}}{Senior Principal Computational Biologist}{F. Hoffmann-La Roche AG}{Basel, Switzerland}{}{I develop algorithms and
software to mine, interpret, model and integrate heterogeneous data, and apply
mathematical and computational tools to support preclinical drug discovery
projects. I am responsible for target assessment and validation, multiscale
modelling of drug mechanism and safety, and MoA characterisation and de-risking
of drug candidates. With colleagues I co-develop novel platforms and resources
to support drug discovery, for instance
the~\href{https://en.wikipedia.org/wiki/Molecular_phenotyping}{molecular
phenotyping} platform and the Small-molecule PAthway Research Kit (SPARK)
library.}

\section{Teaching, mentoring, and academic commitment}
\cvitem{2018--\hspace{4.25ex}}{Teaching lecture series~\textit{Introduction to
\textbf{A}pplied \textbf{M}athematics and \textbf{I}nformatics In \textbf{D}rug
\textbf{D}iscovery} (\url{http://AMIDD.ch}) and~\textit{\textbf{M}athematical
and \textbf{C}omputational \textbf{B}iology in \textbf{D}rug \textbf{D}iscovery}
(\url{http://MCBDD.ch}), Department of Mathematics and Informatics, University
of Basel.}
\cvitem{2018--\hspace{4.25ex}}{Participation in the lecture
series~\textit{From Novel Targets To Novel Therapeutic Modalities}, Master
Programme Drug Sciences, University of Basel.}
\cvitem{2016--\hspace{4.25ex}}{Supervising
master students in bioinformatics and computational biology, and co-supervise
PostDocs in collaboration with academic and industrial collaboration partners.}
\cvitem{2012--\hspace{4.25ex}}{Reviewing research proposals for funding agencies
and manuscripts for journals including \emph{Bioinformatics}, \emph{PLOS Comp.
Biol.}, \emph{NAR Genom. and Bioinform.}, \textit{etc.}}
\cvitem{2012--\hspace{4.25ex}}{Leading courses, seminars and
workshops about programming, bioinformatics, and computational biology in drug
discovery. Certificated as a Carpentury Instructor in 2021.}

\section{Education}
\cventry{2008--2011}{Dr.rer.nat. Bioinformatics}{German Cancer Research Center/ Universit\"at Heidelberg}{}{}{Computational and statistical approaches to study gene networks, supervised by Dr. Stefan Wiemann}
%I have also been working with image analysis of high-content microscopy.}
\cventry{2007-2007}{Marie-Curie Fellow}{Huber Group, European Institute of Bioinformatics}{}{Cambridge, UK}{}
\cventry{2006--2008}{M.Sc. Bioinformatics}{Universit\"at Heidelberg}{Germany}{}{}
\cventry{2002--2006}{B.Sc. Biology, hon.}{Peking University}{Beijing, China}{}{}  % arguments 3 to 6 are optional
% \cventry{1999--2002}{High-school diplom}{Tianjin Nr.1 High School}{Tianjin, China}{}{}  % arguments 3 to 6 are optional
%% B.Sc.: Mit Auszeichnung, Ranking top 5\%. Nebenfach: Germanistik, Ranking 1/30
%\section{Promotion}
%\cvline{Thema}{\emph{}}
%\cvline{Betreuer}{PD Dr. Stefan Wiemann, Abteilung Molekulare Genomanalyse, DKFZ}
%\cvline{Thesis Advisory Commitee}{Dr. Wolfgang Huber, Dr. Rainer K\"onig}
%\cvline{Kurzbeschreibung}{\small}

% Publications from a BibTeX file
\nocite{*}
\renewcommand\refname{Selected peer-reviewed publications}
\bibliographystyle{plainyr-rev}
\bibliography{mypubs-selected}       % 'publications' is the name of a BibTeX file
\cvitem{}{$\to$ See the full, manually curated list at \href{http://goo.gl/CoeJu7}{Google Scholar} (\url{http://goo.gl/CoeJu7}).}

\section{Selected open-source software}
\cvitem{BESCA}{Single-cell omics data analysis pipeline, built together with BEDA colleagues.}
\cvitem{BioQC}{Detecting tissue heterogeneity in high-throughput expression
data.}
\cvitem{KEGGgraph}{Data mining and network analysis
  of biological pathways as graphs in R and Bioconductor.}

%\section{Talks}
%\section{Vortr\"age}
%\cventry{November 2008}{RpsiXML: Bridging PSI-MI XML standard with statistical and computational environment of R}{Workshop on the Development of Standards-compliant tools for molecular interaction data management, Cambridge, UK}{}{}{}

%\section{Poster and publications without review (selected)}
%\cventry{November 2010}{Catching the dynamics of RNAi phenotypes
%  during a human kinome screen}{System Genomics 2010, Heidelberg,
%  Germany}{\textbf{JD. Zhang}, Udo Eichenlaub, \textit{et al.}}{}{}
%\cventry{June 2009}{A novel large-scale screen to identify modulators
%  of miR-21}{MicroRNA and Cancer Meeting 2009, Colorado,
%  USA}{\textbf{JD. Zhang}, I. Keklikoglou, \textit{et al.}}{}{}
%\cventry{February 2009}{Analyzing the Effect of Gene Knock-Down in
%  Real-Time Using Roche's xCELLigence System}{Biochemica
%  2009/02}{\textbf{JD. Zhang}, A. Duda, \textit{et al.}}{}{}
%\cventry{February 2009}{qPCR Identification of Genes Involved in Apoptosis and Cell Cycle Regulation}{Biochemica 2009/02}{\textbf{JD. Zhang}, U. Ernst, A. Irsigler, S. Wiemann and U. Tschulena}{}{}
%\cventry{Mai 2008}{DNAdeconvolutor: software for DNA histogram analysis in R and Bioconductor}{Systems Genomics Meeting 2008, Heidelberg}{\textbf{JD. Zhang}, S. Bechtel, S. Wiemann and A. Poustka}{}{}
%\cventry{August 2007}{Protein Interaction Pool (PIP) -- A comprehensive database and interface to protein interaction data}{ISCB 2007, Wien}{J. Kunert, T. Beissbarth, \textbf{JD. Zhang}, A. Poustka, S. Wiemann and A. Mehrle}{}{}

\renewcommand{\listitemsymbol}{-} % change the symbol for lists
%\section{Current projects}
%\section{Aktuelle Projekte}
%\cvlistitem{Impact of microRNA regulation on biological pathways}
%\cvlistitem{Cell-based screening to identify regulators of human microRNA miR-21 (Kooperation mit Dr. Ulrich Tschulena)}
%\cvlistitem{New cell-based screening with multi-demensional output to identify novel regulators of ERBB signaling system (Kooperation mit Dr. \"Ozg\"ur Sahin)}


\section{Computer skills}
\cvcomputer{Programming}{R/Bioconductor, C/C++, Java, Python}{Database}{PostgreSQL, SQLite, MongoDB}%{Datenbank}{MySQL, MSSQL}
\cvcomputer{Scripting}{Python, Bash, Erlang}{Web}{HTML, CSS, JavaScript, Flask, FastAPI}
\cvcomputer{OS}{Debian Linux, Windows}{Others}{d3js, OpenCV, ImageMagick, \LaTeX{}}
%\cvcomputer{category 3}{XXX, YYY, ZZZ}{category 6}{XXX, YYY, ZZZ}

%\cventry{2003--\hspace{4.25ex}}{Volunteering Board of the association \emph{GerFans China}}{}{}{}{}% arguments 3 to 6 are optional
%\cventry{2004--2006}{Reporter and editorial assistant specialized in Bundesliga}{KICKER China}{Beijing}{}{}
%%Kolumne \emph{Die brillianten 40 Jahre der Bundesliga} für die Yangtze Abendzeitung.}% arguments 3 to 6 are optional
%\cventry{2002--2006}{Deutsch-\"Ubersetzer für chinesische Zeitschriften über Klassische Musik}{}{}{}{}% arguments 3 to 6 are optional

\vspace*{-1em}
\section{Personal information}
\cvitem{Date of Birth}{Sept. 28$^{th}$, 1983 in Tianjin, China}
\cvitem{Marital status}{Married, father of two daughters}
\cvitem{Hobbies}{Family, run \& bike, reading, mathematics, programming, music}
\cvitem{Languages}{English, German, Chinese}
\end{document}

%\renewcommand{\listitemsymbol}{-} % change the symbol for lists

%\section{Extra 1}
%\cvlistitem{Item 1}
%\cvlistitem{Item 2}
%\cvlistitem[+]{Item 3}            % optional other symbol

%\section{Extra 2}
%\cvlistdoubleitem[\Neutral]{Item 1}{Item 4}
%\cvlistdoubleitem[\Neutral]{Item 2}{Item 5}
%\cvlistdoubleitem[\Neutral]{Item 3}{}





%% end of file `template_en.tex'.
