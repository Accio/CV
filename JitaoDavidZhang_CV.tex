%% file `template_en.tex‘, modern cv
%% Copyright 2006-1008 Xavier Danaux (xdanaux@gmail.com).
%
% This work may be distributed and/or modified under the
% conditions of the LaTeX Project Public License version 1.3c,
% available at http://www.latex-project.org/lppl/.


\documentclass[11pt,a4paper]{moderncv}

% moderncv themes
\moderncvstyle{classic} % style options are 'casual' (default), 'classic', 'banking', 'oldstyle' and 'fancy'
\moderncvcolor{blue} % color options 'black', 'blue' (default), 'burgundy', 'green', 'grey', 'orange', 'purple' and 'red'

% character encoding
\usepackage[utf8]{inputenc}                   % replace by the encoding you are using
\usepackage{pslatex}
\usepackage{textcomp}

% font loading
% for luatex and xetex, do not use inputenc and fontenc
% see https://tex.stackexchange.com/a/496643
\ifxetexorluatex
  \usepackage{fontspec}
  \usepackage{unicode-math}
  \defaultfontfeatures{Ligatures=TeX}
  \setmainfont{Latin Modern Roman}
  \setsansfont{Latin Modern Sans}
  \setmonofont{Latin Modern Mono}
  \setmathfont{Latin Modern Math}
\else
  \usepackage[utf8]{inputenc}
  \usepackage[T1]{fontenc}
  \usepackage{lmodern}
\fi

% document language
\usepackage[english]{babel}  % FIXME: using spanish breaks moderncv

% adjust the page margins<
\usepackage[scale=0.75]{geometry}
%\setlength{\hintscolumnwidth}{3cm}						% if you want to change the width of the column with the dates
%\AtBeginDocument{\setlength{\maketitlenamewidth}{6cm}}  % only for the classic theme, if you want to change the width of your name placeholder (to leave more space for your address details
\AtBeginDocument{\recomputelengths}                     % required when changes are made to page layout lengths
\makeatletter
\def\url@leostyle{%
  \@ifundefined{selectfont}{\def\UrlFont{\sf}}{\def\UrlFont{\small\rmfamily}}}
\makeatother

%% Now actually use the newly defined style.
\urlstyle{leo}

% handle two biblatex files
\usepackage[defernumbers=true, sorting=ydnt]{biblatex}
\addbibresource{mypubs-first-last.bib}       % 'publications' is the name of a BibTeX file
\addbibresource{mypubs-not-first-last.bib}       % 'publications' is the name of a BibTeX file
\addbibresource{myposters.bib}       % 'publications' is the name of a BibTeX file

% personal data
\firstname{Jitao David}
\familyname{Zhang}
\title{\fontsize{12}{14}\selectfont A Computational Biologist in Drug Discovery}               % optional, remove the line if not wanted
\address{F. Hoffmann-La Roche AG}{4070 Basel, Switzerland}    % optional, remove the line if not wanted
% \address{Department of Mathematics and Computer Sciences, University Basel}{4070 Basel, Switzerland}    % optional, remove the line if not wanted
\email{jitao\_david.zhang@roche.com}
\homepage{jdzhang.me}

\renewcommand*{\bibliographyitemlabel}{[\arabic{enumiv}]}

%\nopagenumbers{}                             % uncomment to suppress automatic page numbering for CVs longer than one page


%----------------------------------------------------------------------------------
%            content
%----------------------------------------------------------------------------------
\begin{document}
\renewcommand*{\namefont}{\fontsize{32}{32}\sffamily\mdseries\upshape} 
%\renewcommand{\namefont}{\fontsize{30}{32}\sffamily\mdseries\upshape}
\maketitle

\section{Key facts}

\cvlistitem{14+ years of experience in computational biology and drug discovery.}
\cvlistitem{Contributed to drug candidates reaching clinical trials and the market.}
\cvlistitem{Developed and transferred cutting-edge tools for translational science.}
\cvlistitem{Founded and led interdisciplinary research teams with strategic impact.}
\cvlistitem{Coordinated portfolio support and innovation across 60+ experts.}
\cvlistitem{Published 50+ peer-reviewed papers and filed multiple patents.}
\cvlistitem{Lecturer, mentor, and certified vocational trainer.}
\cvlistitem{Invited speaker and active contributor to scientific communities.}
\cvlistitem{Collaborates across Roche, academia, and industry to advance research.}

\section{Professional experience}

\cventry{2022--\hspace{4.25ex}}{Expert scientist}{F. Hoffmann-La Roche AG}{Basel}{Switzerland}{Supporting portfolio projects by developing and applying emerging technologies, validating and developing human-relevant model systems, and designing and implementing computational tools to integrate knowledge and data. Highlights include critical support for entry into human and clinical trials of several drug candidates including Zosurabalpin (currently tested in Phase 3 clinical trials) and Selnoflast (Phase 1), research with Open Models and protein turnover and applications in target and modality selection, as well as development of computational and experimental workflows to profile efficacy and safety profiles of drug candidates using (phospho)proteomics and chemical probes.}
\cventry{2021--\hspace{4.25ex}}{Computer-science apprenticeship trainer and exam expert}{}{}{Switzerland}{Training vocational apprentices specializing in application development and system programming. I co-host qualification examinations as a certified expert.}
\cventry{2018--\hspace{4.25ex}}{Lecturer}{University Basel of Basel}{Basel}{Switzerland}{Teaching two graduate- and post-graduate-level courses at Department of Mathematics and Computer Science, \href{http://amidd.ch}{\emph{Applied Mathematics and Informatics in Drug Discovery}} (AMIDD, 2 credit points) and \href{http://mcbdd.ch}{\emph{Computational and Mathematical Biology in Drug Discovery}} (MCBDD, 2 credit points), as well as contributing to other lecture series.}
\cventry{2018-2022}{Senior principal scientist}{F. Hoffmann-La Roche AG}{}{}{Multiscale modelling of drug mechanism and safety, application of PACE (Pathway Annotated Chemical Ensemble, formerly known as SPARK) library for screening, development of machine-learning empowered toxicity screening assays. Highlights include the TeraTox assay, the identification of RepSox as a tool compound for ophthalmology, de-risking molecules in development, and informing key decisions for several projects.}
\cventry{2016-2018}{Principal scientist}{F. Hoffmann-La Roche AG}{}{}{Developing the molecular phenotyping platform, developing the chemogenomic library SPARK, and supporting projects in the area of infectious diseases. Highlights include a pilot study of molecular phentoyping, identification of EGF update as an early readout for nephrotoxicity, and development of the BioQC software.}
\cventry{2013-2016}{Senior scientist}{F. Hoffmann-La Roche AG}{}{}{Developing the molecular phenotyping platform, supporting Phase IV study of Oseltamivir/Tamiflu (the IRIS study), and applying multi-omics data analysis for disease modelling. Highlights include identification of the time-dependent critical role TGF-beta signaling in breast cancer metastasis, and genomic analysis of the molecular neuropathology of tuberous sclerosis using a human stem cell model.}
\cventry{2011-2013}{Scientist}{F. Hoffmann-La Roche AG}{}{}{Profiling compound efficacy and safety with omics methods, data mining, and machine learning. Highlights include the identification of a predictive gene network for drug safety by analyzing the TG-GATEs database, identification of the causal role of the JAK/STAT3 pathway in turning white adipose to brown, and informing decisions in multiple drug-discovery projects.}
\cventry{2008--2011}{Dr.rer.nat. Bioinformatics}{German Cancer Research Center and Universit\"at Heidelberg}{Heidelberg}{Germany}{Thesis: Computational and statistical approaches to study gene networks, supervised by Prof. Dr. Stefan Wiemann}
\cventry{2007}{Marie-Curie Fellow}{European Institute of Bioinformatics}{}{Cambridge, UK}{}
\cventry{2006--2008}{M.Sc. Bioinformatics}{Universit\"at Heidelberg}{Germany}{}{}
\cventry{2006--2008}{Research associate}{German Cancer Research Center}{Heidelberg}{Germany}{Bioinformatics algorithm and software development in the HUSAR Bioinformatics group}
\cventry{2005--2006}{Journalist and associate editor}{Kicker China}{Beijing}{China}{Translating German articles on European football leagues and report German football league (German: \textit{Bundesliga}) for Chinese readers}
\cventry{2002--2006}{B.Sc. Biology, hon.}{Peking University}{Beijing, China}{}{With a minor in German language and culture}

\section{Founding, leading, and serving teams}

\cvitem{2024--\hspace{4.25ex}}{Currently I assume the role of Portfolio Lead for cardiovascular metabolism, immunology, infectious disease, ophthalmology of the Predictive Modeling and Data Analytics team. I coordinate project support and innovation activities with input from more than 60 experts with a wide range of expertise in computational biology, biostatistics, translational modeling and simulation, disease modeling, and pharmacometrics. I work closely with other Portfolio Leads to present the team in communities and decision boards. Our teamwork has made critical impact on dozens of portfolio projects ranging from entry into portfolio to clinical trials and filing.}
\cvitem{2024--2025}{I served as a core team member of the PADCo (PBPK-ADME-DPL) community, organizing and facilitating meetings, exchanges, and activities of the community, which consists of more than 100 experts in modelling and simulation, drug absorption, distribution, metabolism, and excretion, translational drug metabolism-pharmacokinetics-pharmacodynamics, and clinical pharmacology.}
\cvitem{2022--2024}{I co-founded and lead a new research group within my organization, with 6 permanent positions, 2 postdocs, and interns. The team consists of computational scientists with training or working experience in toxicology, chemistry, biology, and statistics. I led the interviews and the on-boarding of new employees. I am responsible for integrating the team in the working environment, embedding the team in the process of drug discovery and development, and defining the team's vision, strategy, and scientific focus. The team has grown to become an indispensable unit in our organization.}
\cvitem{2022--2023}{I led a cross-functional team of about 30 people to improve the productivity and efficiency with regard to data, insight, and knowledge management. Stakeholders and team members gave positive feedback to the organization and the product of the team. The outcome has led to changes in how data and computational models are captured and used.}

\section{Supervision and mentoring}

\subsection{Postdoctoral researchers}

\cvitem{Dr. Silvan Käser}{2025- (co-supervision with Dr. Neil Parrott, Dr. Nenad Manevski, and Dr. Michael Reutlinger), predicting ADME/PK properties of emerging drug modalities}
\cvitem{Dr. Davide Bassani}{2023-2024 (co-supervision with Dr. Neil Parrott and Dr. Nenad Manevski), computational approaches to preclinical pharmacokinetic property prediction}
\cvitem{Dr. Milad Adibi}{2020-2022 (co-supervision with Dr. Ekaterina Breous-Nystrom), multiscale modeling of drug-induced liver toxicity, currently senior computational biologist at University of Zurich}
\cvitem{Dr. Simon Gutbier}{2018-2020 (co-supervision with Dr. Christoph Patsch and Dr. Markus Britschgi), immune pathway characterization with tool-compound screening for Parkinson's Disease, currently principal scientist at Roche}

\subsection{Student interns}

\cvitem{Alessio D'Addio}{2025 internship working on knowledge and data integration for target and modality selection}
\cvitem{Alexandra Valeanu}{2023, internship working on inferring causal kinases with phosphoproteomics data}
\cvitem{Simon Crouzet}{2022, internship working on predicting protein-ligand interaction with graph neural networks, currently PhD student in Dal Peraro's group at EPFL}
\cvitem{Anja Lieberherr}{2022-2023 (co-supervision with Prof. Niko Beerenwinkel at ETH), master thesis on applying graph neural networks in drug discovery, currently junior technology consultant at BearingPoint, Zürich}
\cvitem{Sarah Morillo}{2021-2022, internship working on proteomics data and protein half life, currently PhD student in Erik van Nimwegen's group at Biozentrum, University of Basel}
\cvitem{Andreea Ciuprina}{2020 (co-supervision with Prof. Niko Beerenwinkel), internship and master thesis on computational inference of immune-cell contribution to drug-induced liver toxicity, currently data engineer and scientist at Endress+Hauser Flowtec}
\cvitem{Rudolf Biczok}{2018-2019 (co-supervision with Prof. Alexandros Stamatakis, KIT), internship and master thesis on building a gene-expression database and evaluating gene-set comparison metrics, currently full-stack Java architect and Azure cloud specialist at Bank for International Settlements (BIS)}
\cvitem{Moaraj Hasan}{2017-2018, internship working on automating gene-expression data analysis}
\cvitem{Tao Fang}{2017-2018 (co-supervision with Prof. Mark Robinson, UZH), internship and master thesis on histopathology prediction with deep neural networks, currently postdoctoral fellow in von Mering group at University of Zurich}
\cvitem{Gregor Sturm}{2016-2017, internship working on gene signatures and the BioQC software, currently clinical bioinformatics scientist, Boehringer Ingelheim}

\subsection{Vocational apprentices specializing in computer sciences}

\cvitem{Nina Lareida}{2024-2025, currently working with Roche}
\cvitem{Niklas Trapp}{2024-2025, currently working with Roche}
\cvitem{Giulia Ferraina}{2023-2024, currently part-time working at Roche and studying computer science at FHNW}
\cvitem{Jannick Lippuner}{2023-2024, currently working with Roche}
\cvitem{Yannik Benndorf}{2022-2023, currently studying computer sciences in TU München}
\cvitem{Paul Geser}{2022-2023, currently studying AI in Hochschule Luzern (HSLU). I am Mr. Geser's Roche mentor as part of the \emph{Study With Roche} offer that sponsors the study.}
\cvitem{Lukas Rinke}{2021-2022, currently studying Data Sciences at FHNW. I am Mr. Rinke's Roche mentor as part of the \emph{Study With Roche} offer that sponsors the study.}
\cvitem{Leonardo Seminatore}{2021-2022, currently bachelor of science in geomatics, FHNW}

\section{Outreaching}

\subsection{Teaching activities}

\cvitem{2020--\hspace{4.25ex}}{Courses on causal inference and computational toxicology for the PSI (Statisticians in the Pharmaceutical Industry) community, once a year}
\cvitem{2019--\hspace{4.25ex}}{Lecture series~\textit{\href{http://MCBDD.ch}{Mathematical and Computational Biology in Drug Discovery}} at the University of Basel, 2 hours per week in the spring semester}
\cvitem{2018--\hspace{4.25ex}}{Lecture series~\textit{\href{http://AMIDD.ch}{Introduction to Applied Mathematics and Informatics In Drug Discovery}} at the University of Basel, 2 hours per week in the fall semester}
\cvitem{2018--\hspace{4.25ex}}{Lecture series~\textit{From Novel Targets To Novel Therapeutic Modalities}, Master Programme Drug Sciences, University of Basel, one lecture per semester}
\cvitem{2012--\hspace{4.25ex}}{Leading company-internal courses, seminars and workshops about computational biology in drug discovery, on average 2 hours per week}

\subsection{Selected invited talks}

\cvitem{2025}{\textit{Defining Human Dosing for Covalent Inhibitors with Translational PK/PD and Protein Turnover Data}, co-presenting with Neil John Parrot, DMDG Meeting \textit{Focusing on Targeted Covalent Drugs: how to best identify and develop Target Covalent Inhibitors}, Basel, Switzerland}
\cvitem{2024}{\textit{Towards intelligent benefit-risk assessment of drug candidates}, Swiss Society of Pharmacology and Toxicology (SSPT) Spring Meeting 2024, Bern, Switzerland}
\cvitem{2023}{\textit{Finding hope in a hopeless time - How Predictive Modeling and Data Analytics shifts our perspectives about antimicrobial discovery}, workshop on the promise of artificial intelligence to antibacterial drug discovery, 7th AMR Conference, Basel, Switzerland}
\cvitem{2022}{\textit{Towards causal modelling of drug-induced toxicity for preclinical to clinical translation}, the Third In Silico Toxicology meeting, online}
\cvitem{2021}{\textit{Optimization of the TeraTox assay for preclinical teratogenicity assessment}, co-presentation with Manuela Jaklin, OpenTox Virtual Conference, online}
\cvitem{2019}{\textit{Bioinformatics and exploratory data analysis in drug discovery: an industrial perspective}, ISMB/ECCB, Basel, Switzerland}
\cvitem{2018}{\textit{Mathematics in drug discovery: a practitioner’s view}, Perlen-Kolloquium, University of Basel, Switzerland}

\subsection{Voluntary engagement in education}

\cvitem{2023--\hspace{4.25ex}}{Invited member of selection committees for tenure track positions in universities}
\cvitem{2022--\hspace{4.25ex}}{Co-organizer of Roche PMDA (Predictive Modeling and Data Analytics) Summer Schools for PhD students, a one-week event taking place annually}
\cvitem{2021--\hspace{4.25ex}}{Member of Parent's Council at the Erlensträsschen primary school, Riehen, Basel, Switzerland}
\cvitem{2021--\hspace{4.25ex}}{Certified vocational trainer (German: \textit{Lehrmeister}) and exam expert for vocational apprenticeship in computer sciences}
\cvitem{2020--\hspace{4.25ex}}{Certified Software and Data Carpentry instructor, hosting company internal courses on programming and data analysis}
\cvitem{2014--\hspace{4.25ex}}{Invited reviewer for research proposals by national funding agencies}
\cvitem{2012--\hspace{4.25ex}}{Reviewer of scientific manuscripts for journals including \emph{Bioinformatics}, \emph{PLOS Comp. Biol.}, \emph{NAR Genom. and Bioinform.}, \textit{etc.}}

%%%%% publications %%%%%

\renewcommand\refname{Publications with (co-)first or correspondence authorship}
\nocite{*}
\printbibliography[keyword=primary]

\renewcommand\refname{Other peer-reviewed publications}
\nocite{*}
\printbibliography[keyword=secondary]

\section{Publications in other forms}

\subsection{Essays}

\cvitem{2024}{\href{https://www.nature.com/articles/d41586-024-01102-8}{How young people benefit from Swiss apprenticeships}, in \emph{Nature}}

\subsection{Patents}

\cvitem{2020}{Oligonucleotides for modulating RTEL1 expression}
\cvitem{2018}{Pyrrolo[2,3-b]pyrazine compounds as cccDNA inhibitors for the treatment of Hepatitis B Virus (HBV) infection}

\subsection{Book chapters}

\cvitem{2016}{Applied Biclustering Methods for Big and High-Dimensional Data Using R, Chapman \& Hall/CRC Biostatistics Series, edited by Adetayo Kasim, S. H., Ziv Shkedy, Sebastian Kaiser \& Talloen, W.}

\subsection{Selected open-source software}
\cvitem{BESCA}{Single-cell omics data analysis pipeline, built together with BEDA colleagues}
\cvitem{BioQC}{Detecting tissue heterogeneity in high-throughput expression data}
\cvitem{ddCt}{A pipeline to collect, analyse and visualize qRT-PCR results}
\cvitem{HBVouroboros}{A Snakemake-based workflow for hepatitis B virus (HBV) drug discovery}
\cvitem{KEGGgraph}{Data mining and network analysis of biological pathways as graphs}
\cvitem{Kinex}{Inference of causal serine/threonine kinases from phosphoproteomics data}
\cvitem{ribios}{A collection of R packages for computational biology tasks in drug discovery}
\cvitem{RTCA}{Analysis and visualization of data from Roche(R) xCELLigence RTCA systems}

% Publications from a BibTeX file

%\renewcommand\refname{Conference posters and papers}
%\nocite{*}
%\printbibliography[keyword=posters]

\subsection{Preprints (not peer-reviewed)}

\cvitem{2023}{Valeanu \textit{et al.} Kinex infers causal kinases from phosphoproteomics data}
\cvitem{2022}{Geser and Zhang. Gene symbol recognition with GeneOCR}
\cvitem{2021}{Zhang \textit{et al.} A Novel, Anatomy-Similar in Vitro Model of 3D Airway Epithelial for Anti-Coronavirus Drug Discovery}
\cvitem{2019}{Fang \textit{et al.} Gene-set enrichment with regularized regression}

\renewcommand{\listitemsymbol}{-} % change the symbol for lists

\vspace*{-1em}
\end{document}

%\renewcommand{\listitemsymbol}{-} % change the symbol for lists

%\section{Extra 1}
%\cvlistitem{Item 1}
%\cvlistitem{Item 2}
%\cvlistitem[+]{Item 3}            % optional other symbol

%\section{Extra 2}
%\cvlistdoubleitem[\Neutral]{Item 1}{Item 4}
%\cvlistdoubleitem[\Neutral]{Item 2}{Item 5}
%\cvlistdoubleitem[\Neutral]{Item 3}{}

%% end of file `template_en.tex'.
